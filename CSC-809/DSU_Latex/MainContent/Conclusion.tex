%\chapter{\leavevmode Conclusion}
% \chapter*{Conclusion}
% \addcontentsline{toc}{chapter}{Conclusion}
\chapter{\leavevmode Conclusion}
\label{chap:conclusion}

Securing supply chains for critical infrastructure and any production environment is a growing concern. Third-parties or nation state level attackers have been shown to target employees or infrastructure indirectly to gain access to their target's network. One of the devices being examined to aid this research is the SNBC BTP-S80, a USB/serial connected thermal printer. These devices are made with foreign software and hardware, and they are used off the shelf without any security review. In some instances, the devices implement an MPU/MCU and an FPGA for I/O processing, which creates a potential gap for data to be modified. 

Data will be collected by assessing a serial printer and its components, firmware, network stack library, featured capabilities (e.g., intended operations like ESC/POS commands to printed paper over serial), and security protections. Using that information the firmware will be modified in-memory or by reflashing to implement HID cloning for scripted attacks. By demonstrating that the original functionality can be maintained in addition to the attack channel, the research will present a viable supply chain attack that could be used against vendors. The proposed research aims to assess a serial printer for such risks and demonstrate that they could be used as a part of a viable supply chain attack.



% The goal of this research is to assess different serial printers for their equipped hardware for input/output, architecture, memory type/capacity, processor; software for operating system, network stack library (e.g., think Treck TCP vulnerabilities or FreeRTOS specific), featured capabilities (e.g., intended operations like ESC/POS commands to printed paper over serial), and security protections. Then, with the information that has been gathered, make several determinations: what hardware/software protections can be relaxed, if modified how can firmware be pushed to the device, can memory be removed and reflashed, does the device have hardware debugging, is the hardware debugging enabled, is the native operating system open source, can we modify the operating system and save versions with debug symbol data, is the flash storage on the device enough for multiple functions (e.g., webserver and printing or BadUSB and printing).

% Although the research itself is not entirely novel or critical to furthering the field, it proposes an original combination of theory and applied research to PoS security. Using a quantitative approach with cross-sectional surveys for a small sample of printer devices, this research will provide a final report to aide future research and development of a design artifact. Specifically, using the surveyed data to aide the research of a modified RTOS, or similar OS, to create a BadUSB-like device.