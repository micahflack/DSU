\section{Methodology}

There are several parts to the research methodology. First, technical information and datasheets were collected for each of the identified devices. Then, device capabilities were verified before beginning device teardown and flash recovery. During the disassembly, each component was documented and further technical information was gathered from respective manufacturers. The format for presenting the collected data is described later in section \ref{datacollectionprocess}.

\subsection{Research Approach} \label{researchapproach}

For this research, the quantitative approach and survey research was used \autocite{babbie2017basics,creswell2017research}. Because the goal of the research was to gather and examine, point-in-time, data across a sampled population of serial printer devices. By using quantitative survey research, it was possible to evaluate which devices are vulnerable to the attacks hypothesized, as well as, which devices are the most eligible for future design artifact research (i.e., creation of modified OS for HID cloning).


\subsection{Cross-Sectional Survey} \label{casestudy}

Using cross-sectional surveys \autocite{creswell2017research} has multiple benefits. It can be used to represent data as it is taken, rather than over a long period.  The study method also focuses on providing summaries that describe the patterns and context between collected data, and how it relates to the research questions.


\subsection{Data Collection Process} \label{datacollectionprocess}

The data collection process began with gathering technical specifications from device manufacturers. Typically, these contain information about the capabilities of the intended device functions. For a printer, this could contain information ranging from hardware specifications (e.g., CPU, architecture, memory) to things like printed pages per minute. This information forms the baseline for the device survey. Afterward, further specifications were gathered for components as each device was disassembled and examined. Roughly, the types and format of gathered device specifications appear as shown in Table \ref{table:device_specs} (e.g., SNBC BTP-S80 is used here).

\begin{table*}
  \centering
  \caption{Device specifications for SNBC BTP-S80}
  \label{table:device_specs}%
  \begin{tabular}{|p{4cm}|p{12cm}|}
    \hline\rowcolor{gray!30}

    \textbf{Specifications} &  \\
    \hline

    Max print speed & 120mm (Two-Color), 150mm (Grayscale), 250mm (Mono) \\
    \hline

    Printing method & Direct Thermal \\
    \hline

    Paper roll type & 9 x 7, 82.5 x 80 x 57.5mm \\
    \hline

    Bar code support & UPC-A, UPC-E, EAN8, EAN13, Code39, Code93, CODE128, CODABAR, ITF, PDF417, QR Code, Maxicode \\
    \hline

    Printer interpreter & ESC/POS \\
    \hline

    Interfaces & Serial+USB+Ethernet \\
    &  USB+Parallel \\
    &  USB+Serial \\
    &  USB+Bluetooth \\
    &  USB+WiFi \\
    &  USB Only \\
    \hline

    Supported OS & 32-bit (Windows XP/2000/POSReady) \\
    &  64bit (Windows XP/Server 2012) \\
    &  32/64bit (Windows 10/8.1/8/7/Server 2008/Server 2003/Vista) \\
    &  Other (Linux/OPOS/BYJavaPOS Windows/BYJavaPOS Linux) \\
    \hline

    Development Kit & Android, iOS \\
    \hline

    Data Buffer & Receive Buffer RAM: 64KB \\
     &  RAM Bitmap: 128KB \\
     &  Flash Bitmap: 512KB \\
    \hline

    Power Supply & AC 100 $\sim$ 240V, 50/60 Hz Adapter \\
    \hline

    Current/Power Usage & 2.0A / 60W \\
    \hline

    Safety and EMI & FCC/UL \\
    \hline

  \end{tabular}
\end{table*}

Following the previous example, the next step in the data collection process would be identifying the SoC. If there is no beforehand knowledge, the SoC can be identified by comparing gathered datasheets during the components discovery. This is easily accomplished using an online service like FindChips or AllDataSheets \autocite{FindchipsElectronicPart}. The process for gathering flash/memory chip specifications was similar; identify the serial number and manufacturer, then find the component datasheet. Gathering the pin layouts and format is also useful for later stages, should the manual flash need to be recovered.

% The expected type and format for SoCs are described in Table \ref{table:soc_specs}.

% The expected format for memory chips can be seen in Table \ref{table:memory_specs}.

% \begin{table*}
%   \centering
%   \caption{SoC technical specs example using Stellaris LM3S2793 Microcontroller}
%   \label{table:soc_specs}%
%   \begin{tabular}{|p{4cm}|p{12cm}|}
%     \hline\rowcolor{gray!30}

%     \textbf{Specifications} &  \\
%     \hline

%     Architecture & 32-bit ARM \\
%     \hline

%     Platform & ARM Cortex-M3 \\
%     \hline

%     Frequency & 80-MHz, 100DMIPS performance \\
%     \hline

%     Memory & 128KB single-cycle Flash memory \\
%      & 64KB single-cycle SRAM \\
%     \hline

%     Firmware & Internal ROM loaded with StellarisWare \\
%     \hline

%     Advanced Comm. Interfaces & UART, SSI, I2C, I2S, CAN \\
%     \hline

%     Debug Interfaces & JTAG, SWD \\
%     \hline

%     Package format & 100-pin LQFP \\
%     & 108-ball pin BGA \\
%     \hline

%   \end{tabular}
% \end{table*}

% \begin{table*}
%   \centering
%   \caption{Memory specifications example using Infineon Technologies S25FL064P \autocite{S25FL064PSeriesFlash}}
%   \label{table:memory_specs}%
%   \begin{tabular}{|p{4cm}|p{12cm}|}
%     \hline\rowcolor{gray!30}

%     \textbf{Specifications} &  \\
%     \hline

%     Single power supply operation & 2.7 to 3.6V \\
%     \hline

%     Software Features & SPI Bus Compatible Serial Interface \\
%     \hline

%     Memory architecture & Uniform 64KB sectors \\
%     & 256 byte page size \\
%     \hline

%     Programming & Page programming (up to 256 bytes) \\
%     & Operations are page-by-page basis \\
%     & Accelerated mode via 9V W\#/ACC pin \\
%     & Quad page programming \\
%     \hline

%     Erase commands & Bulk erase function \\
%      & Sector erase for 64KB sectors \\
%      & Sub-sector erase for 4KB and 8KB sectors \\
%     \hline

%     Protections & W\#/ACC pin used with Status Register Bits to protect specified memory regions and configure parts as read-only; One time programmable area for permanent and secure identification \\
%     \hline

%     Package format & 16-pin SO \\
%     & 8-contact WSON \\
%     & 24-ball BGA, 5x5 pin config \\
%     & 24 ball BGA, 6x6 pin config \\
%     \hline

%   \end{tabular}
% \end{table*}

% A final report will be created detailing each of these tables for the devices and their identified core components. Operating system features and protections will be loosely summarized for each device, there is no set reporting format or requirements. Using the final report will aid in the process of designing an artifact for future research and testing.

\subsection{Hardware Assessment} \label{hardwareassessment}

NIST SP 800-115 \autocite{NISTSP8001152020} provides general guidelines for performing information security testing and assessment, however, there is little information regarding hardware reverse engineering and firmware analysis. Their guidelines are aimed more towards single/multi-tasking operating systems like Windows or Unix-like, those where network logging and listener agents are feasible. For the targeted devices in this research proposal, a different approach is needed that evaluates the hardware protections of the SoC and flash memory. 

Analysis of device components, once disassembled, requires using a hardware debugger tool with the correct interface. The majority of the targeted devices use joint test action group (JTAG) or single wire debugging (SWD) headers. By referring to the manufacturer datasheet for a given SoC, it was possible to identify the pin layout for serial debugging access.

\begin{figure}[ht]%
  \centering
  \includegraphics[keepaspectratio]{Figures/JTAGExample.png}
  \caption{JTAG pin out example for Texas Instruments LM3S2793}%
  \label{fig:jtag_pinout}%
\end{figure}

Figure \ref{fig:jtag_pinout} is an example showing what the physical SoC looks like on a PCB compared to the pin layout described in the datasheet. The dot in the top left of the SoC denotes the beginning of the pin layout. Counting in a counter-clockwise method indicates the pin number and the associated functions. For instance, to access the JTAG debug interface on the LM3S2793:

\begin{itemize}
  \item TDO: pin 77
  \item TDI: pin 78
  \item TMS: pin 79
  \item TCK: pin 80
  \item GND: pin 82
  \item V\textsubscript{DD}: pin 68
\end{itemize}

Using this information, a device like the JTAGULATOR \autocite{JTAGulator2023} can be connected and enumerate or verify pin layouts as described. Ball joint SoCs require a different process and are much harder to debug if there is no visible header available on the board. Once an interface is connected, if debugger access is not disabled, the researcher can interact with the bootloader to further investigate enabled protections and recover flash storage.

If the JTAG is disabled, the researcher would attempt to recover flash manually using a device like the Segger J-Link \autocite{SEGGERJLinkDebug}. The Segger has pre-defined and existing support for working with flash memory and flash breakpoints. Whereas, using OpenOCD with the JTAGULATOR would require time crafting custom configurations. Assuming there are no access protections to the flash memory, the researcher could begin performing firmware analysis to identify the operating system or potential vulnerabilities. Documenting the size and address range of memory regions was a key part of the process.