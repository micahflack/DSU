%\chapter{\leavevmode Conclusion}
% \chapter*{Conclusion}
% \addcontentsline{toc}{chapter}{Conclusion}
\chapter{\leavevmode Conclusion}
\label{chap:conclusion}

The goal of this research is to assess different serial printers for their equipped hardware for input/output, architecture, memory type/capacity, processor; software for operating system, network stack library (e.g., think Treck TCP vulnerabilities or FreeRTOS specific), featured capabilities (e.g., intended operations like ESC/POS commands to printed paper over serial), and security protections. Then, with the information that has been gathered, make several determinations: what hardware/software protections can be relaxed, if modified how can firmware be pushed to the device, can memory be removed and reflashed, does the device have hardware debugging, is the hardware debugging enabled, is the native operating system open source, can we modify the operating system and save versions with debug symbol data, is the flash storage on the device enough for multiple functions (e.g., webserver and printing or BadUSB and printing).

Although the research itself is not entirely novel or critical to furthering the field, it proposes an original combination of theory and applied research to PoS security. Using a quantitative approach with cross-sectional surveys for a small sample of printer devices, this research will provide a final report to aide future research and development of a design artifact. Specifically, using the surveyed data to aide the research of a modified RTOS, or similar OS, to create a BadUSB-like device.